\frontmatter 

% title page, list of tables, list of figures
% T I T L E   P A G E
% -------------------
% This file goes along with the master LaTeX file uw-ethesis.tex
% Last updated May 27, 2009 by Stephen Carr, IST Client Services
% The title page is counted as page `i' but we need to suppress the
% page number.  We also don't want any headers or footers.
\pagestyle{empty}
\pagenumbering{roman}

% The contents of the title page are specified in the "titlepage"
% environment.
\begin{titlepage}
        \begin{center}
        \vspace*{1.0cm}

        \Huge
        {\bf Electrical and Computer Engineering (ECE) Systems Programming and Concurrency ECE252 Laboratory Manual}

        \vspace*{1.0cm}

        \normalsize
        by \\

        \vspace*{1.0cm}

        \Large
        Yiqing Huang \\
        Jeff Zarnett

        \vspace*{3.0cm}

        \normalsize
        Electrical and Computer Engineering Department \\
        University of Waterloo \\ 

        \vspace*{2.0cm}
\makeatletter
        Waterloo, Ontario, Canada, \@date \\
\makeatother
        
      
        
        \vspace*{1.0cm}

        \copyright\ Y. Huang and J. Zarnett 2019 \\
        \end{center}
\end{titlepage}

% The rest of the front pages should contain no headers and be numbered using Roman numerals starting with `ii'
\pagestyle{plain}
\setcounter{page}{2}

\cleardoublepage % Ends the current page and causes all figures and tables that have so far appeared in the input to be printed.
% In a two-sided printing style, it also makes the next page a right-hand (odd-numbered) page, producing a blank page if necessary.
%\newpage

\cleardoublepage
%\newpage

% T A B L E   O F   C O N T E N T S
% ---------------------------------
\tableofcontents
\cleardoublepage
%\newpage

% L I S T   O F   T A B L E S
% ---------------------------
\listoftables
\addcontentsline{toc}{chapter}{List of Tables}
\cleardoublepage
%\newpage

% L I S T   O F   F I G U R E S
% -----------------------------
\listoffigures
\addcontentsline{toc}{chapter}{List of Figures}
\cleardoublepage
%\newpage

% L I S T   O F   S Y M B O L S
% -----------------------------
% \renewcommand{\nomname}{Nomenclature}
% \addcontentsline{toc}{chapter}{\textbf{Nomenclature}}
% \printglossary
% \cleardoublepage
% \newpage

% Change page numbering back to Arabic numerals
\pagenumbering{arabic}



%Brief overview of the document, acknowledgement, disclaimer
\chapter{Preface}

\section*{Who Should Read This Lab Manual?}
This lab manual is written for students who are taking the Systems Programming and Concurrency course ECE252 at the University of Waterloo.

\section*{What is in This Lab Manual?}

The first purpose of this document is to provide the descriptions of each laboratory project.
The second purpose of this document is to provide a quick reference guide to the relevant development tools 
for completing laboratory projects.

This manual is divided into three parts. 

Part I describes the lab administration policies 

Part II describes a set of course laboratory projects as follows:

\begin{itemize}
    \item Lab1: Introduction to systems programming in the Linux computing environment
    \item Lab2: Multi-threaded concurrency programming with blocking I/O
    \item Lab3: Inter-process communication and concurrency control
    \item Lab4: Parallel web crawling
    \item Lab5: Single-threaded concurrency programming with asynchronous I/O  
\end{itemize}

Part III is a quick reference guide to the Linux software development tools. We will be using the Ubuntu 18.04 LTS operating system. The material in this part needs to be self-studied before the labs start.
The main topics are as follows.

\begin{itemize}
    \item The Linux hardware environment
    \item Editors
    \item Compiler 
    \item Debugger 
    \item Automated builds
    \item Version control
\end{itemize}


% A C K N O W L E D G E M E N T S
% -------------------------------

\section*{Acknowledgments}

We are grateful that Professor Patrick Lam shared his ECE459  projects with us. Eric Praetzel has provided continuous IT support, which makes the Linux computing environment available to our students.

We would like to sincerely thank our students who took the ECE254 and ECE459 courses in the past few years. They provided constructive feedback every term to make the manual more useful, in order to address problems that students would encounter when working on each lab assignment. 

%%% Local Variables:
%%% mode: latex
%%% TeX-master: "main_book"
%%% End:
