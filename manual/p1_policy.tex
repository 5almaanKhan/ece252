\chapter*{Lab Administration Policy }

\section*{Group Lab Policy}

    \begin{itemize}
    \item {\bf Group Size.} All labs are done in groups of {\em two}. 
        A size of three is only considered in a lab section that has 
        an odd number of students, and only one group is allowed to have
        a size of three. All group of three requests are processed on
        a first-come, first-served basis. 
        A group size of one is not permitted except if your group
        splits up. There is no workload reduction if you do 
        the labs individually. 
        Everyone in the group normally gets the same mark.
        The Learn system (\verb+http://learn.uwaterloo.ca+)
        is used to sign up for groups.
        %The Course Book System at URL\\
        %\verb+ https://ecewo32.uwaterloo.ca/cgi-bin/WebObjects/CourseBook+\\
        %is used to signup for groups and reserve lab demo times. 
        {\em The lab group signup is due by 10:00pm on the Second Friday of
        the academic term}. Late group sign-up is not accepted and will
        result in losing the entire lab sign-up mark, which is \verb+2%+
        of the total lab grade. Grace days do not apply to Group Signup.
	       
    \item {\bf Group Split-up.} 
            If you notice a workload imbalance in your group, try to solve it as soon as possible 
            within your group. As a last resort, you can split up your group. 
            Group split-up is only allowed once. You are allowed to join a one member group
            after the split-up. You are not allowed to split up the newly formed group.
            There is a one grace day deduction penalty to be applied to each member of the old group. 
            A copy of the code and documentation that was completed before the group split-up 
            will be given to each individual in the group.
            We highly recommend that everyone stays with their groups as much as possible,
            since the ability to do work as a team will be an important skill in your future career.
            Please choose your lab partners carefully. 
            
    \item {\bf Group Split-up Deadline.} 
        To split up your group starting from a particular lab, 
        you need to notify the lab instructor in writing and 
        sign the group split-up form (see Appendix). 
        For Lab{\em n} ({\em n=1,2,3,4,5}), the group split-up form needs to 
        be submitted to the lab instructor by 4:30pm on Thursday in the week that Lab{\em n} has 
        scheduled lab sessions. 
        If you are late to submit the split-up form, 
        then you need to finish Lab{\em n} as a group and submit 
        your split-up form during the week where Lab({\em n+1}) 
        has scheduled sessions and split starting from Lab({\em n+1}).
    \end{itemize}
    
\begin{table}
\begin{center}
\begin{tabular}{|p{4cm}|l|l|l|}
\hline
Deliverable	  & Weight  & Lab Session Week   & Post-Lab Deliverable Deadline \\ \hline
Group Sign-up &	\verb+2%+	 & N/A      & 22:00 Friday in Week 2     \\ \hline
% % % % % The following is for Spring term	
LAB 1   &	\verb+18%+    & Week 3       & 22:00  2 days after the lab session  \\ \hline
LAB 2   &	\verb+20%+    & Week 5       & 22:00  2 days after the lab session  \\ \hline	
LAB 3   &  	\verb+20%+    & Weeks 8      & 22:00  Sunday after the lab session  \\ \hline
LAB 4   &	\verb+20%+    & Week 10      & 22:00  Sunday after the lab session  \\ \hline
LAB 5   &   \verb+20%+    & Week 12      & 22:00  Sunday after the lab session  \\ \hline
% % % % % The following is for Fall term that starts on a Thursday!!!!
%LAB1   &	    \verb+8%+    & Week 0 and 1       & 10:00pm Wednesday in Week 2  \\ \hline
%LAB2   &	    \verb+30%+   & Weeks 2, 3, 4 and 5       & 10:00pm Wednesday in Week 6  \\ \hline	
%LAB3   &    	\verb+30%+	 & Weeks 7 and 9      & 10:00pm Wednesday in Week 10 \\ \hline
%LAB4   &	    \verb+30%+   & Week 11         & 10:00pm Wednesday in Week 12 \\ \hline
%%LAB5   &	    \verb+30%+   & Week 11         & 10:00pm Tuesday in Week 12 \\ \hline
%Deliverable	  & Weight  & Due Date            & File Name \\ \hline
%Group Sign-up &		    & 04:30pm May. 09th   &             \\ \hline	
%LAB1   &	\verb+2%+	& 11:59pm May. 13th   & LAB1\_Gid.zip \\ \hline
%LAB2   &	\verb+2%+   & 11:59pm May. 27th   & LAB2\_Gid.zip \\ \hline	
%LAB3   &	\verb+6%+	& 11:59pm Jun. 26th   & LAB3\_Gid.zip \\ \hline
%LAB4   &	\verb+4%+   & 11:59pm Jul. 08th   & LAB4\_Gid.zip \\ \hline
%LAB5   &	\verb+6%+   & 11:59pm Jul. 22nd   & LAB5\_Gid.zip \\ \hline
\end{tabular}
\caption{Project Deliverable Weight of the Lab Grade, Scheduled Lab Sessions and Deadlines.}
\label{tb_deadline}
\end{center}
\end{table}
\section*{Lab Assignments Grading and Deadline Policy} 
Labs are graded by lab TAs based on the rubric listed in each lab. The weight of each lab towards your final lab grade is listed in Table \ref{tb_deadline}. 

    \begin{itemize}
    \item {\bf Lab Assignment Preparation and Due Dates.}
        Students are required to prepare for the lab well 
        before they come to the schedule lab session. 
        {\em Pre-lab deliverables for each lab are due by the scheduled lab session starting time}.
        During the scheduled lab session, we either provide 
        in-lab help or conduct lab assignment evaluation 
        or do both at the same time. 
        
        %Post-lab deliverables are normally due on Wednesdays in the week after the scheduled lab session. 
        The detailed deadlines of post-lab deliverables 
        are displayed in Table \ref{tb_deadline}. Note that the reading/study week is not counted as a week in the table. 
        %Please note that different lab sections have different deadlines.
        
        %Table \ref{tb_deadline} lists the Post-lab deliverable deadlines.
        %{\em Please be advised that lab sessions during the midterm week 
        % are cancelled.}
    \item {\bf Lab Assignment Late Submissions.} 
    	Late submission is accepted within five days after the deadline of the lab. 
    	No late submission is accepted five days after the lab deadline.
    	There are five grace days \footnote{Grace days are calendar days. Days in weekends are counted.}
        that can be used for some post-lab deliverables late submissions
        \footnote{A post-lab deliverable that does not accept a late submission 
        will be clearly identified in the lab assignment description. 
        Labs whose evaluation requires demonstrations do not accept late submissions
        of the code.}. 
        A group split-up will consume one grace day. 
        After all grace days are consumed, a \verb+10%+ per day late submission penalty will be applied.
        However, if it is five days after the lab deadline, no submission is accepted. 
        %automatically get a grade of zero.
        
        %You will get a bonus mark of {\em N\%}of your entire lab mark, 
        %where {\em N} is the number of grace days you do not use. 
        %When you use up all your grace days, 
        %a 10\% per day late penalty will be applied to a late submission. 
        %Please be advised that to simplify the book-keeping, 
        %late submission is counted in a unit of day rather than hour or minute. 
        %An hour late submission is one day late, 
        %so does a fifteen hour late submission.
    \item {\bf Lab Re-grading.}
    	To initiate a re-grading process, contact the grading TA in charge first. The re-grading is a rigid process. The entire lab will be re-graded. Your new grades may be lower, unchanged or higher than the original grade received. If you are still not satisfied with the grades received after the re-grading, escalate your case to the lab instructor to request a review and the lab instructor will finalize the case.
    \end{itemize}
\section*{Lab Repeating Policy}
For a student who repeats the course, labs need to be re-done with a new lab partner. 
Simply turning in the old lab code is not allowed. 
We understand that the student may choose a similar route to the solution chosen last time 
the course was taken. However it should not be identical. Since the lab is being done for the second time,
we expect that the student will improve upon the older solution. Also the new lab partner should be 
contributing equally, which will also lead to differences in the solutions. 

Note that the policy is course specific, and at the discretion of the course instructor and the lab instructor.

\section*{Lab Assignments Solution Internet Policy} 
It is not permitted to post your lab assignment solution source code or lab report on the internet freely for the public to access. For example, it is not acceptable to host a public repository on GitHub that contains your lab assignment solutions. A warning with instructions to take the lab assignment solutions off the internet will be sent out upon the first offence. If no action is taken by the offender within twenty-four hours, then a lab grade of zero will automatically be assigned to the offender.
    
\section*{Seeking Help Outside Scheduled Lab Hours}
\begin{itemize}
    \item{\bf Discussion Forum.}
      We recommend that students use the Piazza discussion forum to ask the teaching team questions instead of sending individual emails to the teaching staff.
      For questions related to the current lab project, our target response time is one business day
      \footnote{Our past experiences show that the number of questions spike when deadline is close.The teaching staff will not be able to guarantee one business day response time when workload is above average, though we always try our best to provide a timely response.}. 
    {\em There is no guarantee on the response time to questions about labs other than the current one}.
    \item{\bf Office Hours.} 
    The Learn system calendar gives the office hour details.
    \item{\bf Appointments.}
    Students can also make appointments with the lab teaching staff if their problems are not resolved by the discussion forum or during office hours. The appointment booking is by email.
    
    To make the appointment efficient and effective, when requesting an appointment, 
    please specify three preferred times and roughly how long 
    the appointment needs to be. On average, an appointment is fifteen minutes per project group. 
    In your email that requests the appointment, please also summarize the main questions to be asked.
    If a question requires teaching staff to look at a code fragment, please bring a laptop with the necessary development software installed. 
    
    Please note that teaching staff will not debug a student's program for them. Debugging is part of the exercise of completing a programming assignment. Teaching staff will be able to demonstrate how to use the debugger and provide case-specific debugging tips. Teaching staff will not give a direct solution to a lab assignment. Guidances and hints will be provided to help students to find the solution by themselves.
    
\end{itemize}
%\begin{table}
%\begin{center}
%\begin{tabular}{|l|l|p{5cm}|l|}
%\hline
%Time (even weeks only)  & Location   & Name                     & email ID \\ \hline
%TBA    & DC-2631    & Yiqing (Irene) Huang     & yqhuang\\ \hline
%TBA    & E5-5122        & Anas Abognah &  aabognah \\ \hline
%TBA 10:30-11:30 & E5-4121       & Jean-Christophe Petkovich   & j2petkovich\\ \hline
%\end{tabular}
%\caption{Bi-weekly Office Hour Schedule}
%\label{tb_office_hour}
%\end{center}
%\end{table}

\section*{Lab Facility After Hours Access Policy} 
After hour access to the lab will be given to the class 
when we start to use the Keil boards in lab. 
However, please be advised that after hours access is a privilege. 
Students are required to keep the lab equipment and furniture 
in good condition in order to maintain this privilege. 

No food or drink is allowed in the lab. 
Please be aware that you may be sharing the lab with other classes.  
When resources become too tight, cooperation is required 
such as taking turns using the stations in the lab. 		

%%% Local Variables:
%%% mode: latex
%%% TeX-master: "main_book"
%%% End:
